\section{Меры защиты от группировки}

\subsection{Общие рекомендации}
Некоторые наиболее общие и эффективные методы предотвращения целенаправленной атаки с использованием программы-вымогателя \cite{bib:cybelangel}:

\begin{enumerate}
    \item Регулярно устанавливайте патчи безопасности. В частности, касающиеся уязвимости Bluekeep (уязвимость в реализации Microsoft Remote Desktop Protocol, позволяющая осуществить удалённое выполнение кода) ссылки на RDP-атаки: CVE-2019-0708 и CVE-2019-1181.
    \item По возможности выключайте RDP или используйте очень сложные пароли, так как злоумышленники используют сервисы брутфорса, доступные в даркнете.
    \item Фильтруйте почту по наличию в содержимом исполняемых файлов.
        По возможности блокируйте все письма с исполняемыми файлами в них.
    \item Всегда храните проверенный бэкап наиболее критических данных.
    \item Выключите макросы в пакете Microsoft Office. Особенно для внешнего доступа.
    \item Используйте Windows Access Control, чтобы предотвратить попытки пользователей запускать ПО из локальных папок, таких, как Local App Data; Temp, ProgramData и т.д.
    \item По возможности изолируйте критические серверы.
    \item Разработайте план и политики информационной безопасности.
    \item Проводите регулярные учебные мероприятия с сотрудниками с целью повышения уровня культуры информационной безопасности в коллективе.
\end{enumerate}

\subsection{MITRE D3FEND}

В соответствии с техниками, представленными в MITRE D3FEND \cite{bib:mitre-defend}, с учётом методов, используемых REvil, для обнаружения, изолирования и предупреждения заражения можно также выборочно рекомендовать

\begin{enumerate}
    \item \textit{Использование приманок}.
        Расположение в системе объектов-приманок (D3-DO).
        В частности, в виде сессионных токенов (D3-DST), аутентификационных данных пользователей (D3-DUC) и файлов (D3-DF).
    \item \textit{Анализ запущеных и готовящихся к запуску процессов}.
        Проведение анализа процессов в системе (D3-PA).
        Тщательное наблюдение за запуском процессов, включая аргументы и атрибуты запуска (D3-PSA).
        Определение неавторизованных процессов.
        Анализ системных вызовов и настройка правил их фильтрации на уровне ядра (D3-SCA).
        Проведение попыток обнаружения самомодификации процессов (D3-PSMD).
        Отслеживание изменения кода программы во время выполнения.
        Проверка подписи исполняемых файлов (D3-EAL).
    \item \textit{Мероприятия по обеспечению безопасности интернет-соединения}.
        Фильтрация трафика (D3-OTF), анализ метаданных запросов (D3-PMAD), проверка сертификатов (D3-CA), анализ URL (D3-UA).
    \item \textit{Анализ файлов}.
        Проведение динамического анализа ПО, запуск программ в контроллируемых условиях (D3-DA).
\end{enumerate}
