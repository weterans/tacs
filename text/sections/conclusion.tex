\conclusion

В отчёте рассмотрена деятельность киберпреступной группировки REvil.
Установлен хронологический порядок задержания членов группировки и обстоятельств задержания.
Рассмотрены основые используемые хакерами тактики и техники, такие, как:

\begin{itemize}
    \item внедрение кода в запущенные в системе процессы для повышения привелегий;
    \item запуск процессов от чужого имени с целью сокрытия вредоносного программного обеспечения;
    \item использование вредоносного кода, написанного с использованием языка программирования Visual Basic for Applications с целью запуска кода на автоматизированном рабочем месте жертвы.
\end{itemize}
и другие.

Представлены основные методы и способы защиты от целенаправленных атак, проводимых группировкой.
В их числе:

\begin{itemize}
    \item регулярная установка обновлений и исправлений системы безопасности используемого программного обеспечения с целью предотвращения новых выявленных уязвимостей;
    \item корректная настройка групповых политик для исключения возможности использования техник от имени пользователя;
    \item проведение учебных мероприятий с целью поддержания должного уровня технической грамотности рабочего персонала для самостоятельного предупреждения и противостояния фишинговым атакам.
\end{itemize}
и другие.
