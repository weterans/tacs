\section{Самые крупные проекты, целевая аудитория группировки}

В личном кабинете злоумышленник может контролировать основные параметры атаки и вести переписку с жертвой.
Вирус-вымогатель, имеющий аналогичное название --- REvil, был впервые идентифицирован 17 апреля 2019 года.
Группировка REvil считается подгруппой финансово мотивированной хакерской группировки GOLD SOUTHFIELD.
Эту операцию по распространению вирусов-вымогателей связывают с Россией, из-за того, что она не нацелена на российские организации или организации в странах советского блока.
Также наблюдалась схожесть методов REvil с DarkSide --- другой хакерской преступной группировкой, связываемой с Россией: код программы-вымогателя, используемый DarkSide, напоминает код, используемый REvil, а записки с требованием выкупа составлены почти одинаково, они представлены на рисунках \ref{fig:ransomware-01} и \ref{fig:ransomware-02}.

\begin{figure}[h!]
    \centering
    \includegraphics{ransomware-01.png}
    \caption{Записка с требованием выкупа DarkSide.}
    \label{fig:ransomware-01}
\end{figure}

\begin{figure}[h!]
    \centering
    \includegraphics{ransomware-02.png}
    \caption{Записка с требованием выкупа REvil.}
    \label{fig:ransomware-02}
\end{figure}

Шифровальщик REvil известен тем, что требует у своих жертв рекордно большие выкупы.
В случае отказа от выплаты выкупа операторы REvil выкладывали в открытый доступ конфиденциальную информацию на странице под названием <<Happy Blog>>, которая продемонстрирована на рисунке \ref{fig:revil-website}.
Сайт представлен в доменной зоне \texttt{.onion} в сети Tor.
Адреса в этой сети считаются анонимными, они не являются полноценными записями DNS, и информация о них не хранится в корневых серверах DNS.
Получить доступ к сайтам в доменной зоне .onion можно, посылая запрос через сеть Tor-серверов.

\begin{figure}[h!]
    \centering
    \includegraphics{revil-website.png}
    \caption{Сайт блога группировки REvil.}
    \label{fig:revil-website}
\end{figure}

Вирус использовался против организаций в производственном, транспортном и электрическом секторах. 

Получила известность атака шифровальщика REvil на тайваньскую компанию Quanta Computer в апреле 2021 года.
Это крупнейший в мире производитель ноутбуков, а также одна из немногих компаний, которая собирает продукты 
Apple на основе предоставленных им дизайнов и схем. Выкуп составлял 50 миллионов долларов.
В случае отказа операторы REvil угрожали опубликовать в открытом доступ схемы и чертежи компонентов MacBook.
Однако ни Apple, ни Quanta не стали публично реагировать на заявление хакеров и не пошли с ними на переговоры.
Некоторые схемы действительно были опубликованы на сайте REvil, но их подлинность доказана не была.
В итоге все упоминания о взломе и опубликованные схемы были удалены.

В мае 2021 года компания JBS S.A. (крупнейший в мире поставщик мяса) подверглась атаке программы-вымогателя, 
ответственность за которую возлагают на группировку REvil. Атака привела к временной остановке всех заводов 
компании по производству говядины в США и нарушила работу птицеводческих и свиноводческих заводов.
Компания JBS заплатила REvil выкуп в размере 11 миллионов долларов, чтобы предотвратить публичную утечку 
украденных данных и устранить возможные технические проблемы.

В июле 2021 года клиенты компании Kaseya Limited, которые пользовались ПО для удалённого управления, стали 
жертвами группировки REvil. Kaseya Limited --- американская компания, которая разрабатывает программное 
обеспечения для управления сетями, системами и инфраструктурой информационных технологий.
В качестве выкупа было потребовано 70 миллионов долларов за восстановление всех зашифрованных данных.
Вскоре после этого группировка REvil исчезла, платежные сайты и инфраструктура были закрыты.
Компания Kaseya заявила, что они получили универсальный дешифратор от «доверенной третьей стороны» и теперь 
распространяют его среди пострадавших клиентов. Также они заявили, что не могут подтвердить или опровергнуть 
факт уплаты выкупа.
