\section{Обнаружение группировки}

Непосредственную причину возможности ареста участников выделить проблемаиично.
В этой связи ограничимся хронологией событий.

13 июля 2021 года сайты REvil в даркнете перестали отвечать на поисковые запросы.
Некоторые эксперты в США предположили, что неожиданное исчезновение REvil из даркнета может быть связано с телефонным разговором между президентами США и России накануне \cite{bib:bbc-1}.

Ведущие зарубежные издания --- New York Times, CNN, BBC, независимый источник новостей и аналитических материалов о кибербезопасности Threatpost и другие --- связали это действие с возможной блокировкой группы американскими спецслужбами, сворачиванием деятельности по приказу российских спецслужб или хакеры просто «ушли в тень», для чего покинули сетевое пространство, чтобы обезопасить себя от возможного ареста, как считают специалисты, в том числе директор по технологиям компании BreachQuest Джейк Уильямс (англ. Jake Williams) \cite{bib:threatpost}.

14 января 2022 года в ходе спецоперации ФСБ и МВД России, проведённой по запросу властей США, деятельность группировки была пресечена.
Задержание проходило на территории Москвы, Санкт-Петербурга, Московской, Ленинградской и Липецкой областях \cite{bib:gazeta}.
У хакеров изъяли 426 миллионов рублей, 500 тысяч евро, 600 тысяч долларов, 20 автомобилей премиум-класса \cite{bib:fsb, bib:lenta, bib:tass}.

Тем не менее, есть основания полагать, что группировка <<вернулась>> и возобновила свою деятельность.
19 апреля ИБ-специалисты pancak3 и Soufiane Tahiri первыми заметили активность сайтов REvil. Дело в том, что новый «сайт для утечек» REvil начал рекламироваться через русскоязычный форум-маркетплейс RuTOR (не путать с одноименым торрент-трекером). Новый сайт размещен на другом домене, но связан с исходным сайтом REvil, который использовался, когда группа еще была в строю. Также на 26 страницах сайта перечислены компании, пострадавшие от рук вымогателей, большинство из которых — старые жертвы REvil. Лишь две последние атаки, похоже, связаны с новой кампанией, а один из пострадавших — нефтегазовая компания Oil India \cite{bib:xaker}.
