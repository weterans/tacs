\introduction

REvil (Ransomware Evil; группировка также известна как Sodinokibi) --- операция, которая заключается в распространении вирусов-вымогателей, используя сервис <<вымогатель как услуга>> (ransomware-as-a-service, RaaS).
Этот сервис предлагает услуги по управлению атаками программ-вымогателей на заказ.
Обычно возможности подобного сервиса включает аренду зловреда-шифровальщика или блокировщика, услуги ботнета для доставки вредоносного ПО и панель управления.
Заработанные средства заказчик и производитель вредоносного ПО делят в заранее установленном соотношении \cite{bib:upguard}.

\begin{figure}[h!]
    \centering
    \includegraphics[width=0.8\textwidth]{cloudwars-statistics.png}
    \caption{Программы-вымогатели в цифрах \cite{bib:cloudwars}.}
    \label{fig:statistics}
\end{figure}

В 2021-м году атаки с использованием программ-вымогателей ударили по 80\% опрошенных организаций \cite{bib:forbes}, среди которых более 60\% заплатили выкуп.
При этом, размер выкупа в 2021 году в среднем увеличился на 40\% по сравнению с 2020 годом \cite{bib:cloudwars}.
