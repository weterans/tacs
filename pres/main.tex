\documentclass{beamer}

\usepackage[utf8]{inputenc}
\usepackage[russian]{babel}

\usepackage{graphicx}
\usepackage{subfig}
\graphicspath{{./figures}}

\usefonttheme{professionalfonts}

\usetheme{EastLansing}
\setbeamertemplate{caption}[numbered]

\title[REvil]{Анализ киберпреступной группировки REvil}
\author[Фирсов, Голуб, Афонин]{Фирсов Г. В., Голуб С. А., Афонин В. Д.}
\date{21.05.2022}

\begin{document}

\maketitle

\begin{frame}{Содержание доклада}
    \tableofcontents
\end{frame}

\section{Общая информация}
\begin{frame}{Введение}
    \begin{figure}[h!]
        \centering
        \includegraphics[width=0.6\textwidth]{revil-logo.jpg}
    \end{figure}
    \begin{itemize}
        \item Название исходит от слов Ransomware Evil
        \item Альтернативное название группировки --- Sodinokibi
        \item Группировку можно считать подгруппой GOLD SOUTHFIELD
        \item Впервые вирус был идентифицирован в апреле 2019
    \end{itemize}
\end{frame}

\begin{frame}{Сходство с DarkSide}
    \begin{itemize}
        \item Kод, используемый DarkSide, напоминает код, используемый REvil
        \item Записки с требованием выкупа составлены почти одинаково
    \end{itemize}
    \begin{figure}[h!]
        \centering
        \includegraphics[width=0.6\textwidth]{ransomware-02.png}
        \caption{Записка с требованием выкупа REvil.}
    \end{figure}
\end{frame}

\begin{frame}{Сходство с DarkSide}
    \begin{figure}[h!]
        \centering
        \includegraphics[width=0.8\textwidth]{ransomware-01.png}
        \caption{Записка с требованием выкупа DarkSide.}
    \end{figure}
\end{frame}

\begin{frame}{Сайт группировки}
    \begin{figure}[h!]
        \centering
        \includegraphics[width=0.8\textwidth]{revil-website.png}
        \caption{Сайт блога группировки REvil.}
    \end{figure}
\end{frame}

\begin{frame}{Атака на Quanta Computer (апрель 2021)}
    \begin{figure}[h!]
        \centering
        \includegraphics[width=0.6\textwidth]{makbok.png}
        \caption{Чертежи компонентов MacBook.}
    \end{figure}
\end{frame}

\begin{frame}{Атака на JBS S.A. (май 2021)}
    \begin{figure}[h!]
        \centering
        \includegraphics[width=0.8\textwidth]{jbs.png}
        \caption{Проценты доходов компании от стран.}
    \end{figure}
\end{frame}

\begin{frame}{Атака на Kaseya (июль 2021)}
    \begin{figure}[h!]
        \centering
        \includegraphics[width=0.8\textwidth]{kaseya.png}
        \caption{Этапы атаки на компанию Kaseya.}
    \end{figure}
\end{frame}

\section{Тактики и техники}
\begin{frame}{Интерпретатор скриптов или оболочка командной строки: Visual Basic (T1059.005)}
    \begin{figure}[h!]
        \centering
        \includegraphics[width=0.95\textwidth]{obfuscated-vba.png}
        \caption{Обфусцированный код на языке VBA.}
    \end{figure}
\end{frame}

\begin{frame}{Внедрение в процессы (T1055)}
    \begin{figure}[h!]
        \centering
        \includegraphics[height=0.7\textheight]{dll-injection.png}
        \caption{Общая схема атаки внедрения DLL.}
    \end{figure}
\end{frame}

\begin{frame}{Прочие тактики и техники}
    \begin{itemize}
        \item Манипулирование маркерами доступа: имперсонация/кража маркера (T1134.001)
        \item Манипулирование маркерами доступа: создание процесса с использованием чужого маркера (T1134.002)
        \item Интерпретатор командной строки: PowerShell (T1059.001)
        \item Пользовательский запуск: вредоносный файл (T1204.002)
        \item ...
    \end{itemize}
\end{frame}

\section{Обнаружение группировки}
\begin{frame}{Арест (январь 2022)}
    14 января 2022 года в ходе спецоперации ФСБ и МВД России, проведённой по запросу властей США, деятельность группировки была пресечена.
    \begin{figure}[h!]
        \centering
        \subfloat[Арест участника группировки.]{
            \centering
            \includegraphics[width=0.45\textwidth]{arrest-01.jpg}
        }
        \subfloat[Суд участника группировки.]{
            \centering
            \includegraphics[width=0.45\textwidth]{arrest-02.jpg}
        }
        \caption{Задержание и суд участников группировки.}
    \end{figure}
\end{frame}

\begin{frame}{Возвращение (апрель 2022)}
    19 апреля ИБ-специалисты pancak3 и Soufiane Tahiri первыми заметили активность сайтов REvil.
    \begin{figure}[h!]
        \centering
        \includegraphics[height=0.6\textheight]{revil-blog.png}
        \caption{Активный сайт блога REvil на момент 21.05.2022.}
    \end{figure}
\end{frame}

\section{Меры защиты от группировки}
\begin{frame}{Общие рекомендации}
    \begin{itemize}
        \item Регулярно устанавливайте патчи безопасности. В частности, касающиеся уязвимости Bluekeep (уязвимость в реализации Microsoft Remote Desktop Protocol, позволяющая осуществить удалённое выполнение кода) ссылки на RDP-атаки: CVE-2019-0708 и CVE-2019-1181.
        \item По возможности выключайте RDP или используйте очень сложные пароли, так как злоумышленники используют сервисы брутфорса, доступные в даркнете.
        \item Фильтруйте почту по наличию в содержимом исполняемых файлов.
        \item Всегда храните проверенный бэкап наиболее критических данных.
        \item Проводите регулярные учебные мероприятия с сотрудниками с целью повышения уровня культуры информационной безопасности в коллективе.
    \end{itemize}
\end{frame}

\begin{frame}{MITRE D3FEND}
    \begin{itemize}
        \item \textit{Использование приманок}.
            \begin{itemize}
                \item сессионные токены (D3-DST)
                \item аутентификационные данные пользователей (D3-DUC)
                \item файлы (D3-DF)
            \end{itemize}
        \item \textit{Анализ запущеных и готовящихся к запуску процессов}.
            \begin{itemize}
                \item Анализ системных вызовов и настройка правил их фильтрации на уровне ядра (D3-SCA).
                \item Проведение попыток обнаружения самомодификации процессов (D3-PSMD).
                \item Проверка подписи исполняемых файлов (D3-EAL)
                \item ...
            \end{itemize}
    \end{itemize}
\end{frame}

\begin{frame}{MITRE D3FEND (продолжение)}
    \begin{itemize}
        \item \textit{Мероприятия по обеспечению безопасности интернет-соединения}.
            \begin{itemize}
                \item анализ метаданных запросов (D3-PMAD)
                \item проверка сертификатов (D3-CA)
                \item анализ URL (D3-UA)
                \item ...
            \end{itemize}
        \item \textit{Анализ файлов}.
            \begin{itemize}
                \item проведение динамического анализа ПО
                \item запуск программ в контроллируемых условиях (D3-DA).
            \end{itemize}
    \end{itemize}
\end{frame}

\section{Заключение}
\begin{frame}{Заключение}
    \textit{Основые используемые хакерами тактики и техники}:
    \begin{itemize}
        \item повышение привелегий
        \item запуск процессов от чужого имени
        \item использование вредоносного кода, написанного на VBA
    \end{itemize}
    и другие.

    \vspace{0.5cm}

    \textit{Основные методы и способы защиты от целенаправленных атак}:
    \begin{itemize}
        \item регулярная установка обновлений и исправлений системы безопасности
        \item корректная настройка групповых политик
        \item проведение учебных мероприятий персонала
    \end{itemize}
    и другие.
\end{frame}

\end{document}
