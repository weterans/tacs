\section{Используемые тактики и техники}

Данный раздел посвящен описанию основных используемых группировкой REvil тактик и техник.

\subsection{Access Token Manipulation: Token Impersonation/Theft (T1134.001)}

Основная суть техники заключается в получении токена (маркера доступа) пользователя ОС Windows
и имперсонации с помощью этого токена. Делается это через следующие функции Win32 API:
\texttt{SetThreadToken}, \texttt{ImpersonateLoggedOnUser}.

Токен же может быть получен разными путями: например, его можно достать из процесса проводника
(explorer.exe), запущенного пользователем. Такой прием был замечен за софтом группировки REvil 
\cite{bib:mcafee}: если ПО запущено от имени системы, то производилось понижение привилегий через 
токен, полученный из процесса проводника, что позволяло снизить вероятность обнаружения вредоносного 
ПО, так как после понижения прав оно уже не могло как-либо повлиять на файловые ресурсы пользователя 
SYSTEM (в Windows это отдельная учетная запись). В данном случае это тактика Defense Evasion.

Мерами защиты будут верно настроенные групповые политики, запрещающие создание токенов обычным 
пользователям и администраторам с неполным токеном. Кроме того, в системе должны быть верно настроены
права учетных записей, то есть права администратора не должны быть даны обычным пользователям.

\subsection{Access Token Manipulation: Create Process with Token (T1134.002)}

Суть техники заключается в запуске процесса от имени иного пользователя, что достигается при помощи
вызова \texttt{CreateProcessWithToken} или \texttt{ShellExecute[Ex]} (используя параметр 
\texttt{runas}). Вообще говоря, в общем случае для создания процесса с чужим токеном может 
потребоваться его имперсонация.

Вредоносное ПО REvil может повышать свои привилегии как раз при помощи запуска \texttt{ShellExecute}
с параметром \texttt{runas} \cite{bib:secureworks}: если программа обнаруживает, что запущена с низкими 
привилегиями, то она запускает свою копию с повышенными привилегиями и завершается. В данном случае 
это тактика Privilege Escalation.

Меры защиты от данной техники совпадают с мерами защиты от Т1134/001 и описаны в разделе 2.1.

\subsection{Command and Scripting Interpreter: PowerShell (T1059.001)}

Техника заключается в использовании оболочки PowerShell в ОС Windows для выполнения различных действий. 
При помощи данной оболочки можно загружать, например, произвольные файлы из интернета, после чего 
выполнять код, записанный в них. 

Стоит отметить, что обращение к исполняемому файлу powershell.exe не является обязательным -- команды 
можно выполнять через .NET интерфейс \texttt{System.Management.Automation}.

В частности, группировкой REvil PowerShell и командная строка (T1059.003) используются удаления теневых 
копий томов, а также для скачивания различных файлов.

\subsection{Command and Scripting Interpreter: Visual Basic (T1059.005)}

Visual Basic -- скриптовый язык программирования. Его производные, такие как VBA, тесно интегрированы
с некоторыми продуктами Microsoft Office. Код скрипта может быть разбит на некоторое количество
файлов с целью усложнения анализа и обнаружения (Defence Evasion). Для дальнейшего усложнения анализа
могут применяться техники обфускации. Относится техника к тактике Execution.

Группировкой REvil используется для скачивания различных файлов и запуска их на машине жертвы 
\cite{bib:gdata}. Пример обфусцированного кода на VBA показан на рис. (\ref{fig:obfuscated-vba}).

\begin{figure}[h!]
    \centering
    \includegraphics[width=\textwidth]{obfuscated-vba}
    \caption{Обфусцированный код на языке VBA.}
    \label{fig:obfuscated-vba}
\end{figure}

Техника тесно связана с другой -- Obfuscated Files or Information (T1027), так как зачастую, как сказано 
выше, с целью усложнения анализа применяются различные методы обфускации кода.

\subsection{Modify Registry (T1112)}

Вредоносное ПО может сохранять в реестре Windows различные конфигурационные данные, регистрироваться для
автоматического запуска при загрузке системы. Для этого обыкновенно используется встроенная в ОС утилита
Reg или специальное API: \texttt{RegCreateKey[Ex]}, \texttt{RegSetValue[Ex]} и иные функции. Может использоваться
для сокрытия от механизмов защиты (Defence Evasion), а также для закрепления в системе (Persistence, хотя
для этого выделяется отдельная техника -- T1547.001).

Так, программа-вымогатель, имеющая отношение к рассматриваемой группировке, создает в реестре ключ
\texttt{HKEY\_LOCAL\_MACHINE\textbackslash SOFTWARE\textbackslash recfg}, в котором сохраняются различные
ключи и иные данные, используемые при шифровании пользовательских файлов \cite{bib:mcafee}. Если запись в 
раздел \texttt{HKEY\_LOCAL\_MACHINE} не удалась, то данные пишутся в раздел \texttt{HKEY\_CURRENT\_USER}.

\subsection{Process Injection (T1055)}

Process Injection -- обширная техника, включающая в себя множество различных подтехник. Активно используется
рассматриваемой группировкой Revil в своих вредоносных программах \cite{bib:mcafee-1}. Суть техники состоит 
во внедрении произвольного кода в существующие процессы и исполнение его в их адресном пространстве. Может 
использоваться как для повышения привилегий при внедрении в привилегированные системные процессы 
(тактика Privilege Escalation) или для сокрытия от защитных механизмов (тактика Defence Evasion).

Внедрить код в существующий процесс можно посредством, например, атаки DLL Injection (T1055.001), которая
основывается на схожести сигнатур функций \texttt{LPTHREAD\_START\_ROUTINE} и \texttt{LoadLibrary}. Так,
в целевом процессе при помощи функции \texttt{VirtualAllocEx} выделяется кусок памяти, в который записывается
(\texttt{WriteProcessMemory}) путь до DLL с целевым кодом, после чего в этом же процессе при помощи функции
\texttt{CreateRemoteThread} создается поток, в качестве функции которого выступает \texttt{LoadLibrary},
загружающая указанную библиотеку и выполняя код, прописанный в \texttt{DllMain}. Помимо этого, выполнить код
в адресном пространстве другого процесса возможно посредством механизма Asynchronous Procedure Call (APC).
При помощи API-функции \texttt{QueueUserApc} или недокументированной \texttt{NtQueueApcThread} в очередь 
пользовательских APC произвольного потока в системе (в частности в другом процессе) произвольную функцию,
код которой предварительно необходимо записать в процесс по той же схеме, что описана выше.


\subsection{Прочие тактики и техники}

Кроме описанных выше группировкой REvil используются также следующие тактики и техники, заслуживающие 
внимания \cite{bib:mitre}:
\begin{itemize}
    \item Application Layer Protocol: Web Protocols (T1071.001) используется для взаимодействия с сервером
        управления (Command and Control).
        
    \item Data Encrypted for Impact (T1486) -- данные на компьютере жертвы шифруются, а для расшифрования
        требуется выкуп (Impact).
        
    \item Encrypted Channel: Asymmetric Cryptography (T1573.002) -- используется для защищенной коммуникации
        с сервером управления (Command and Control). REvil использует алгоритм ECIES.
        
    \item Indicator Removal on Host: File Deletion (T1070.004) -- вредоносный код помечает себя на удаление
        при ближайшей перезагрузке системы, тем самым обнаружить его после перезагрузки будет сильно сложнее
        (Defense Evasion).
        
    \item Masquerading: Match Legitimate Name or Location (T1036	.005) -- вреднонсная программа именуется как
        исполняемый файл для обновления Microsoft Office Word и тем самым не вызывает подозрения у пользователя
        (Defense Evasion).
        
    \item Native API (T1106) -- многие из описанных выше техник используют API ОС Windows. Некоторые функции,
        используемые для отдельных техник упомянуты в соответствующих подразделах. В основном относится к
        тактике Execution, однако может применяться и для других целей.
        
    \item Phishing: Spearphishing Attachment (T1566.001) -- вредоносное ПО группировки REvil распространяется
        с помощью вложений в электронной почте (Initial Access).
        
    \item User Execution: Malicious File (T1204.002) -- тесно связанная с предыдущей тактика. Вредоносное ПО
        запускается на компьютере жертвы через VBA-макросы (Execution), встроенные во вредоносные документы 
        Microsoft Office, распространяемые по электронной почте.
        
    \item Scripting (T0853) -- Revil активно использует скрипты, написанные на языках JavaScript, WbScript, 
        PowerShell (Execution). Так, например, JS-скрипты могут содержать обфусцированные скрипты PowerShell, 
        которые непосредственно запускают вредоносные бинарные файлы.
        
    \item User Execution (T0863) -- упомянутые JS-скрипты запускаются по клику пользователя на вредоносное
        почтовое вложение (Execution).
        
    \item Service Stop (T0881, T1489) и System Service Discovery (T1007) -- вредоноснове ПО REvil при помощи
        API ОС может получать информацию о запущенных в системе процессах и службах, а также останавливать их.
        Техники относятся к тактикам Impact, Inhibit Response Function и Discovery соответственно.
\end{itemize}

Стоит отметить, что этот список не является исчерпывающим для рассматриваемой группировки, однако перечисленные
техники и соответствующие тактики представляют основной интерес. Иные же, не попавшие в данный список, носят 
вспомогательный характер.
